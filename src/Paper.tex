% Template for Elsevier CRC journal article
% version 1.2 dated 09 May 2011

% This file (c) 2009-2011 Elsevier Ltd.  Modifications may be freely made,
% provided the edited file is saved under a different name

% This file contains modifications for Procedia Computer Science

% Changes since version 1.1
% - added "procedia" option compliant with ecrc.sty version 1.2a
%   (makes the layout approximately the same as the Word CRC template)
% - added example for generating copyright line in abstract

%-----------------------------------------------------------------------------------

%% This template uses the elsarticle.cls document class and the extension package ecrc.sty
%% For full documentation on usage of elsarticle.cls, consult the documentation "elsdoc.pdf"
%% Further resources available at http://www.elsevier.com/latex

%-----------------------------------------------------------------------------------

%%%%%%%%%%%%%%%%%%%%%%%%%%%%%%%%%%%%%%%%%%%%%%%%%%%%%%%%%%%%%%
%%%%%%%%%%%%%%%%%%%%%%%%%%%%%%%%%%%%%%%%%%%%%%%%%%%%%%%%%%%%%%
%%                                                          %%
%% Important note on usage                                  %%
%% -----------------------                                  %%
%% This file should normally be compiled with PDFLaTeX      %%
%% Using standard LaTeX should work but may produce clashes %%
%%                                                          %%
%%%%%%%%%%%%%%%%%%%%%%%%%%%%%%%%%%%%%%%%%%%%%%%%%%%%%%%%%%%%%%
%%%%%%%%%%%%%%%%%%%%%%%%%%%%%%%%%%%%%%%%%%%%%%%%%%%%%%%%%%%%%%

%% The '3p' and 'times' class options of elsarticle are used for Elsevier CRC
%% The 'procedia' option causes ecrc to approximate to the Word template
\documentclass[3p,times,procedia]{elsarticle}
\flushbottom

%% The `ecrc' package must be called to make the CRC functionality available
\usepackage{ecrc}
\usepackage[bookmarks=false]{hyperref}
    \hypersetup{colorlinks,
      linkcolor=blue,
      citecolor=blue,
      urlcolor=blue}
%\usepackage{amsmath}


%% The ecrc package defines commands needed for running heads and logos.
%% For running heads, you can set the journal name, the volume, the starting page and the authors

%% set the volume if you know. Otherwise `00'
\volume{00}

%% set the starting page if not 1
\firstpage{1}

%% Give the name of the journal
\journalname{Procedia Computer Science}

%% Give the author list to appear in the running head
%% Example \runauth{C.V. Radhakrishnan et al.}
\runauth{William (Billy) Charlton}

%% The choice of journal logo is determined by the \jid and \jnltitlelogo commands.
%% A user-supplied logo with the name <\jid>logo.pdf will be inserted if present.
%% e.g. if \jid{yspmi} the system will look for a file yspmilogo.pdf
%% Otherwise the content of \jnltitlelogo will be set between horizontal lines as a default logo

%% Give the abbreviation of the Journal.
\jid{procs}

%% Give a short journal name for the dummy logo (if needed)
%\jnltitlelogo{Computer Science}

%% Hereafter the template follows `elsarticle'.
%% For more details see the existing template files elsarticle-template-harv.tex and elsarticle-template-num.tex.

%% Elsevier CRC generally uses a numbered reference style
%% For this, the conventions of elsarticle-template-num.tex should be followed (included below)
%% If using BibTeX, use the style file elsarticle-num.bst

%% End of ecrc-specific commands
%%%%%%%%%%%%%%%%%%%%%%%%%%%%%%%%%%%%%%%%%%%%%%%%%%%%%%%%%%%%%%%%%%%%%%%%%%

%% The amssymb package provides various useful mathematical symbols

\usepackage{amssymb}
%% The amsthm package provides extended theorem environments
%% \usepackage{amsthm}

%% The lineno packages adds line numbers. Start line numbering with
%% \begin{linenumbers}, end it with \end{linenumbers}. Or switch it on
%% for the whole article with \linenumbers after \end{frontmatter}.
%% \usepackage{lineno}

%% natbib.sty is loaded by default. However, natbib options can be
%% provided with \biboptions{...} command. Following options are
%% valid:

%%   round  -  round parentheses are used (default)
%%   square -  square brackets are used   [option]
%%   curly  -  curly braces are used      {option}
%%   angle  -  angle brackets are used    <option>
%%   semicolon  -  multiple citations separated by semi-colon
%%   colon  - same as semicolon, an earlier confusion
%%   comma  -  separated by comma
%%   numbers-  selects numerical citations
%%   super  -  numerical citations as superscripts
%%   sort   -  sorts multiple citations according to order in ref. list
%%   sort&compress   -  like sort, but also compresses numerical citations
%%   compress - compresses without sorting
%%
%% \biboptions{authoryear}

% \biboptions{}

% if you have landscape tables
\usepackage[figuresright]{rotating}
%\usepackage{harvard}
% put your own definitions here:x
%   \newcommand{\cZ}{\cal{Z}}
%   \newtheorem{def}{Definition}[section]
%   ...

% add words to TeX's hyphenation exception list
%\hyphenation{author another created financial paper re-commend-ed Post-Script}

% declarations for front matter


\begin{document}
\begin{frontmatter}

%% Title, authors and addresses

%% use the tnoteref command within \title for footnotes;
%% use the tnotetext command for the associated footnote;
%% use the fnref command within \author or \address for footnotes;
%% use the fntext command for the associated footnote;
%% use the corref command within \author for corresponding author footnotes;
%% use the cortext command for the associated footnote;
%% use the ead command for the email address,
%% and the form \ead[url] for the home page:
%%
%% \title{Title\tnoteref{label1}}
%% \tnotetext[label1]{}
%% \author{Name\corref{cor1}\fnref{label2}}
%% \ead{email address}
%% \ead[url]{home page}
%% \fntext[label2]{}
%% \cortext[cor1]{}
%% \address{Address\fnref{label3}}
%% \fntext[label3]{}

\dochead{The 12th International Workshop on Agent-based Mobility, Traffic and Transportation Models, Methodologies and Applications (ABMTrans 2023)\\ March 15-17, 2023, Leuven, Belgium}%
%% Use \dochead if there is an article header, e.g. \dochead{Short communication}
%% \dochead can also be used to include a conference title, if directed by the editors
%% e.g. \dochead{17th International Conference on Dynamical Processes in Excited States of Solids}

\title{SimWrapper, a web-based open source platform for interactive visualization of transport data and microsimulation outputs}

\author[a]{William Charlton}

\address[a]{Technische Universität Berlin, Chair of Transport Systems Planning and Transport Telematics, Straße des 17. Juni 135, 10623 Berlin, Germany}

\begin{abstract}
YY
Data visualization has been an integral part of transportation planning and travel demand modeling for decades. This effort describes the essential features and use cases of a general travel data visualization platform. A new web-based, open-source, configurable platform is then presented that can produce a wide array of interactive charts, maps, and dashboards that are generally useful in the transportation domain. The details of software design are provided along with several examples of implementation at a public agency. The platform is found to be very flexible and the straightforward configuration using text files enables efficient development and deployment of web-based interactive visualizations.

\end{abstract}

\begin{keyword}
Data visualization; Transport microsimulation; MATSim; ActivitySim
\end{keyword}
\cortext[cor1]{Corresponding author. Tel.: +1-415-335-9282.}
\end{frontmatter}

%\correspondingauthor[*]{Corresponding author. Tel.: +0-000-000-0000 ; fax: +0-000-000-0000.}
\email{charlton@vsp.tu-berlin.de}

%%
%% Start line numbering here if you want
%%
% \linenumbers

%% main text

%\enlargethispage{-7mm}
\section{Main Text}
\label{main}

Having confirmed the utility and capabilities of a fully browser-based
data visualization approach for three individual project portals, we
then set out to generalize the method. An entirely generic data
visualization platform is inherently more difficult than a project
portal, as every researcher investigates widely disparate questions and will be focusing on completely different outputs. Where one researcher may be using MATSim to predict future dynamic-response shared taxi vehicle flows, another is doing emergency-response evacuation planning, or emissions reduction through increased transit ridership efforts. The tool needs to be extremely flexible.

% \begin{nomenclature}
% \begin{deflist}[A]
% \defitem{A}\defterm{radius of}
% \defitem{B}\defterm{position of}
% \defitem{C}\defterm{further nomenclature continues down the page inside the text box}
% \end{deflist}
% \end{nomenclature}


% \begin{figure}
%   \centering
% 	\begin{minipage}{.75\textwidth}
% 		\includegraphics[width=\textwidth]{chapters/06-simwrapper/images/charts.png.pdf}
% 		\caption{CHart: stuff.}
% 		\label{fig:chartychart}
% 	\end{minipage}
% \end{figure}

This chapter describes \textbf{SimWrapper}, an open-source web-based
data visualization platform we developed with the goal that it be useful
for anyone working with MATSim outputs or even outputs from other
data-intensive microsimulation models.


% Bulleted lists may be included and should look like this:
% \begin{itemize}[]
% \item First point
% \item Second point
% \item And so on
% \end{itemize}


% \subsection{ Tables}

% All tables should be numbered with Arabic numerals. Every table should have a caption. Headings should be placed above tables, left justified. Only horizontal lines should be used within a table, to distinguish the column headings from the body of the table, and immediately above and below the table. Tables must be embedded into the text and not supplied separately. Below is an example which the authors may find useful.

% \begin{table}[h]
% \caption{An example of a table.}
% \begin{tabular*}{\hsize}{@{\extracolsep{\fill}}lll@{}}
% \toprule
% An example of a column heading & Column A ({\it{t}}) & Column B ({\it{t}})\\
% \colrule
% And an entry &   1 &  2\\
% And another entry  & 3 &  4\\
% And another entry &  5 &  6\\
% \botrule
% \end{tabular*}
% \end{table}

% %\enlargethispage{12pt}


\section{Overview: SimWrapper, in a nutshell}

Many design questions were already settled in the aforementioned
research YY.

SimWrapper, in a nutshell:

% \begin{itemize}[]
% \item First point
% \item Second point
% \item And so on
% \end{itemize}

\begin{itemize}[]
\item
    is a static website that runs client-side javascript in the form of a ``single page application'', a common approach in current web
    development that is compatible with all modern web browsers;
\item
    supports network-based file storage for public- and/or
    group-accessible shared data files (model runs), but has no other
    back-end server requirements and can run completely locally if no
    network file storage is available or needed;
\item
    allows the user to navigate through their local filesystem or shared
    network storage of model runs to view results that are saved in a
    specific folder, rather than a database-centric approach. This
    matches the design of MATSim and other simulation models which
    produce collections of output files by default;
\item
    provides a collection of data visualization archetypes that are each
    appropriate for displaying a certain type of data, for example
    various statistical chart types (bars, lines, area, pie), geographic
    data viewers supporting road and transit network link data, area
    aggregation (``choropleth'' and ``spider'') maps, XY coordinate
    plots, and many more;
\item
    can combine all of these disparate components into cohesive
    dashboards that the user can lay out in a flexible manner, using
    small declarative configuration files. These configurations can be
    applied across multiple projects or simulation runs;
\item
  is GDPR (General Data Protection Requirement) compliant,
  as it performs no user tracking, has no centralized data storage, no advertising, nor any other privacy-compromising misfeatures.
  SimWrapper is not a product for sale; it is an open research
  platform.
\end{itemize}

The following sections explore the design of SimWrapper in more detail.

\section{Reuse of existing framework and
components}

\cite{CharltonLaudan2020WebBasedVisualization} did it first.


The starting point for SimWrapper was the PAVE project website described
in section YY. This ``single page application'' approach involves
selecting a curated set of javascript infrastructure libraries for
common needs, and then writing bespoke code for our specific use case
and the ``glue'' between the components.

Our experience with PAVE led us to select existing Javascript libraries
for the following:

\begin{itemize}
\item
  User interface interaction: the ``Vue'' framework YY is the primary
  glue that links the page layout with user interactions such as mouse
  clicks, running code when user-initiated events occur
\item
  Data loading: Most MATSim outputs are either tabular text files in CSV
  format, or compressed XML files with explicit schemas. The Papaparse
  and Fast-XML-Parser libraries handle loading these two data formats
\item
  Charting: the PAVE site included statistical charts such as bar, line,
  pie, and scatter plots, and used the Plotly javascript library. Plotly
  is very easy to use but not as feature-rich as some other choices; see
  below YY for updated capabilities
\item
  Geographic data on maps: our initial efforts using the Mapbox
  javascript library led us to the more performant Deck.gl collection of
  map-based visualizations.
\item
  Animation: Three.js is a very flexible 3D animation library that is
  used for PAVE vehicle animation visualizations.
\end{itemize}

All of these libraries share compatible open-source licenses, and are
included in SimWrapper under the terms of those licenses.

\section{Modifications necessary}

Direct user feedback, described in detail in section YY, allowed us to
map out a set of changes and improvements for the generic tool. In
summary, changes were needed in the following categories:

\begin{itemize}
\item
  Performance. The network link viewer in particular was slow to load
  datasets for large simulations. This was not a problem for PAVE or
  AVÖV because the study areas were less populated.
\item
  Flexibility. Each of the data visualization components needed to be
  made much more flexible. For example, the PAVE link viewer assumed
  that input data was specified by time period, whereas a generic tool
  needs to depict any sort of data.
\item
  Output traversal. While PAVE had a hard-coded set of alternatives that
  could be browsed in a simple manner, a generic tool needs some sort of
  model run traversal capability; a way to browse the hierarchical file
  storage available.
\item
  Stability and resilience. The PAVE site included almost no error
  message reporting or helpful debugging infrastructure, because expert
  analysts carefully crafted the inputs for each alternative. A generic
  tool needs to be tolerant of user mistakes and helpful in guiding the
  user when inputs are lost or malformed.
\item
  Better defaults plus configurability. We do not intend to replicate a
  full-featured desktop application, of which there are many in the
  Geographic Information System (GIS) realm. Rather, users expressed a
  desire for a set of clear, curated defaults that have some
  configurability. For much more advanced configuration, a
  professionally-developed package such as QGis is likely more
  appropriate.
\end{itemize}

\subsubsection{Audience}

The PAVE website was intended to be public-facing: both agency staff and
actual members of the public could navigate the site. It presented a
small set of YY six alternatives, depicting the same visualizations for
each alternative.

SimWrapper could be public facing, but is predominantly used in its
current form by researchers and professional analysts at public
agencies.

YY

\section{Accessing files through a web
browser}

The use case of file storage via departmental file server is
well-explored and very functional, as expressed in the project websites
for AVÖV, PAVE, and COVID-Sim.

A key difference between the earlier project websites and SimWrapper is
the need to ``meet the users where they are'' -- in other words, we
cannot rely on there being a departmental file server with a public API
endpoint serving data files. One of the primary feedback elements from
the initial MatHub implementation described in chapter YY was that it
was too onerous to upload model run outputs to a second server system
before being able to view or analyze anything. In addition to being
wasteful of space (and MATSim outputs can be gigabytes in size!), it is
time-consuming.

For regular users in the middle of their research workflow, something
else is needed. Most of our internal users run simulations either on
their personal laptop/desktop machines, or on the university compute
cluster, which has extensive attached storage but no public-facing
access via the web. Furthermore, these runs are often not intended to be
immediately publicized.

Thus we explore several avenues for enabling users to view their
outputs, described here.

% ------------------------------------------------------------------------------------
% ### How SimWrapper access files via HTTP
% ------------------------------------------------------------------------------------

\subsection{How SimWrapper access files via
HTTP}

SimWrapper is designed to allow browsing of files from
administrator-defined HTTP URLs, which represent the root of the file
storage for that project. For example, the PAVE project datasets are all
stored on the VSP public file server at the URL

\url{https://svn.vsp.tu-berlin.de/repos/public-svn/matsim/scenarios/countries/de/berlin/projects/pave/website/}

That URL is the defined ``root'' of the project; all of the project
dashboard configurations, model outputs, and processed data files exist
in various subfolders below that location. The PAVE website at
https://vsp.berlin/pave is set up to read files from that base URL. (But
refer to section YY for a discussion of CORS configuration, which is
necessary to allow one website to read the files stored on another.

SimWrapper elevates this to allow multiple configured root filesystems;
the public VSP file server is one such root, but others can also be
configured and are displayed on the home page of SimWrapper. Each root
is expected to provide HTTP directory access to this storage: SimWrapper
needs to be able to \emph{view directory listings} and \emph{retrieve
file contents}. SimWrapper never writes any files anywhere; it is
read-only.

% ------------------------------------------------------------------------------------
% ### Local files
% ------------------------------------------------------------------------------------

\subsection{Local files on a personal
laptop/desktop}

This design presents a problem for local files: By design, all web
browsers explicitly forbid file-system access from any websites by
default. This default is certainly a good default; no one wants any
random website to start sniffing around their home directory.

But in our case this is not any random website: we \emph{want}
SimWrapper to see the files in some of our local folders. How can this
be accomplished?.

After several explorations including raw HTML files opened directly,
arcane experimental browser flags (always vendor specific!), and other
less fruitful avenues, the one method that consistently works for all
browsers is as follows: for browsing local files on a machine, first
start up a small helper application which is itself a simple HTTP
server. This tiny server responds to HTTP requests and delivers the
directory contents requested. The server listens on ``localhost'',
i.e.~your own computer, generally on port 8000. So the full URL is
\url{http://localhost:8000/}.

Once this is set up and running, this HTTP endpoint can be accessed in
SimWrapper just like any other external file storage. SimWrapper knows
be default to look for files at URL \url{http://localhost:8000}.

As part of this research we wrote a very small Python library which
provides this server. Any machine with Python 3.x installed can run
\texttt{pip\ install\ simwrapper} to install this mini file server, and
then run it by navigating to their data folder and running the command
\texttt{simwrapper\ serve}. That includes all of the server components
and configuration needed to server files to SimWrapper.

Some configuration notes:

\begin{itemize}
\item
  The local HTTP server will only serve the files from inside the
  working directory in which it is started, including any subfolders. No
  other folders on the user's machine are exposed.
\item
  The computer's operating machine has default firewall and router rules
  that will generally prohibit any outside access from other computers
  on the LAN or the Internet. This can be modified; see YY
\item
  Some configuration details for the server that are important for our
  use case: we must enable access from websites at different URLs using
  ``CORS'' configuration headers; see YY
\item
  Some browsers (Safari, and now recent builds of Chrome) sometimes
  block access to localhost sites or http sites (vs.~https sites), see
  discussion at YY
\item
  Every language framework already includes some sort of ``Tiny HTTP
  Server'' library for just these types of uses: in Python it is in the
  \texttt{http.server} library, in Java there is the Jibble
  SimpleWebServer. Our \texttt{simwrapper} python tool leverages the
  existing Python infrastructure.
\item
  We also wrote a java version, published as
  \texttt{mini-file-server.jar} for users who are more comfortable
  running Java-based software than Python.
\end{itemize}

The Python tool including source and user documentation is currently
available at \url{https://pypi.org/project/simwrapper/}.

The Java tool is currently available at
\url{https://github.com/simwrapper/mini-file-server}

% ----------------------------------------------
\subsubsection{Data security and
privacy}

With this setup, the SimWrapper site itself is loaded from the Internet,
but once loaded, the user's data never leaves their computer. SimWrapper
is an entirely client-based system with absolutely no upstream server.
The javascript runs in the users' browser, accessing files available on
localhost only -- also on the user's own computer. Nothing leaves the
browser. If there are privacy or confidentiality issues with model
outputs, SimWrapper can still be used for analysis in this ``localhost''
mode.

% ------------------------------------------------------------------------------------
% ### Files on Compute Cluster servers
% ------------------------------------------------------------------------------------

\subsection{Files residing on the university compute cluster, accessed
via SSH}

This local-http-server paradigm can be extended to access files on any
remote university computer cluster using the SSH (``secure shell'')
protocol.

SSH is usually the protocol (and command) used to log into remote
systems. There is a parallel command which allows ``mounting'' the
remote file system using the SSH protocol. The remote files are mapped
to a folder on the user's system; once mounted, the user can browse the
files inside that folder as if they were local files (but generally more
slowly, depending on network throughput conditions).

\begin{itemize}
\item
  Linux users can install the command \texttt{sshfs} to add this
  capability;

  \begin{itemize}
  \item
    once installed, a command similar to
    \texttt{sshfs\ username@cluster.math.tu-berlin.de:/net/myfiles\ cluster}
    will mount the remote folder \texttt{/net/myfiles} to my local
    folder \texttt{cluster}. You would change the username, URL, and
    folder names to match your situation.
  \item
    \texttt{sudo\ umount\ cluster} or similar to unmount.
  \end{itemize}
\item
  YY MacOS is similar to Linux but requires installing the sshfs fuse
  driver first
\item
  YY Windows users have many options for FUSE-based sshfs support, this
  repo is nice one \url{https://github.com/billziss-gh/sshfs-win}
\end{itemize}

% ------------------------------------------------------------------------------------
% ### Files on other machines: simwrapper here
% ------------------------------------------------------------------------------------

\subsection{Files residing on another machine on the local LAN
network}

A challenging use case presented by users is one or more central
``modeling server'' machines on the local LAN, where most runs are
performed and which contain the simulation outputs.

The aforementioned localhost-based HTTP server does not work in this
situation, because a user sitting at their computer, opening the
Internet-based SimWrapper website, trying to read files served via
localhost on the modeling server, will always be blocked by browser
security measures. After many hours trying to find a way to sneak around
these restrictions, we accepted that this security measure is working as
designed, and we need a different approach.

The reason this approach is blocked is because the SimWrapper website is
hosted on a secure ``HTTPS'' server, while the localhost files must be
served without encryption using HTTP. Setting up an encryption
certificate is difficult because internal LAN machines don't typically
have world-findable DNS entries. This combination of secure and insecure
content is blocked by all browsers.

A workaround is to serve the files and the site from the same file
server, instead of using the Internet-based SimWrapper that is hosted at
vsp.berlin. We are already asking users to run a small file server to
access their local files, thus extending that file server to also serve
the necessary javascript and HTML is a natural extension.

And this is what we have done: a special mode is added to the
\texttt{simwrapper} python tool named \texttt{simwrapper\ here}. Now the
little server will serve both the file contents of the folder in which
it is started, \emph{and} the SimWrapper website itself.

The user runs \texttt{simwrapper\ here} on the file server instead of
\texttt{simwrapper\ serve}, \emph{noting the full URL printed in the
console}, and then on their personal computer browses to that URL
instead of to vsp.berlin/simwrapper. In this manner, the site and any
local files stored on that server are made available, together.

This also implies that SimWrapper can be used completely offline once it
is installed.

% ------------------------------------------------------------------------------------
% ### Special case: Google Chrome
% ------------------------------------------------------------------------------------

\subsection{Special case: Chrome and the ``File System Access
API''}

Google Chrome and a subset of other browsers based on the Chromium
codebase implement an experimental API known as ``File System Access
API''. This is not part of the official Web specification, and it may
never be adopted by other browser vendors.

But for users running Google Chrome, this experimental API provides
another avenue for accessing local files, one which completely
eliminates the need for the local file server approaches above.

This is considered ``progressive enhancement,'' or in other words,
adding features to the website when the browser is identified as
supporting them.

On Chrome, users will see an additional element on the main page of
SimWrapper, a button allowing them to grant access to local files. The
browser will open a standard folder-picking dialog followed by a warning
that granting this permission will allow the SimWrapper site to view the
files in that folder. Et Voíla, that is exactly what we need.

Once permission is granted, local files are immediately visible without
any additional configuration. This permission can be revoked and may be
re-requested every time the browser restarts.

YY show browser grant access dialog

% ------------------------------------------------------------------------------------
% ## Converting purpose-built vizes into Generic Data Viewers
% ------------------------------------------------------------------------------------

\section{Converting purpose-built visualizations into generic data
viewers}

The underlying infrastructure -- the build system, the user interaction
libraries, and the choice of off-the-shelf components -- was more or
less complete after the PAVE, AVÖV, and COVID-Sim projects were
operational. But the specific views needed a great deal of retooling to
make them useful in a generic manner.

This section describes some of the most challenging aspects of this
process of genericizing SimWrapper.

% ------------------------------------------------------------------------------------
% ### Link Viewer
% ------------------------------------------------------------------------------------

\subsection{Network link viewer}

The link viewer was originally scoped to display link volumes only, such
as a typical ``bandwidth plot'' commonly used in travel modeling. Even
for PAVE this was short-sighted, as the project team quickly found other
uses for the viewer such as link-based emissions.

Two critical updates made for SimWrapper are (1) the removal of the
assumption that the data inputs will always have time period data in the
columns, plus a summary ``grand total'' column; and (2) that colors and
widths must be configurable, preferably separately.

These changes are now part of SimWrapper -- see YY for a typical plot.

YY show a bandwidth plot

User testing shows that a great deal more is still necessary to meet
user expectations. Data filters and configurable hovers are two of the
most-requested enhancements.

YY user testing

% ------------------------------------------------------------------------------------
% ### XY Data Plots
% ------------------------------------------------------------------------------------

\subsection{XY Hexagon plots}

Much data is not link-based, even for transport simulations. Activity locations, home locations, pickups and dropoffs for transit and taxi modes, all have geographic coordinates associated with them but are not necessarily attached to specific links.

A new visualization type, the ``XY Hexagons'' plot, depicts these types of data by aggregating them into user-definable hexagonal buckets. The number of points inside the hexagons corresponds to a color or height; this is user-configurable.

YY show XY Hexagon plot

The default MATSim output \texttt{output\_trips.csv} includes this type of data, and is automatically viewable without any configuration at all if it is present in a SimWrapper data folder.

% ------------------------------------------------------------------------------------
% ## Calculation Tables
% ------------------------------------------------------------------------------------

\section{Calculation tables: providing top-line summary metrics}

Early in the development of SimWrapper, user feedback identified the need for reporting basic summary statistics and generating common aggregate values from model runs. These top-level summaries provide the first indication of useful results, such as overall mode share, average travel times, total emissions, and so forth. In addition, these measures are an excellent way to "sanity-check" a model run; in other words, to identify any obvious glaring errors in those topline numbers when compared to previously-established norms.

The first element needed is a straightforward way for users to specify the inputs, outputs, and formatting of calculation tables, compatible with the file-based configuration approaches already in use for the graphical visualizations. A second challenge is to formulate a clear and accurate scheme for specifying the needed calculations, including statistical transformations of the inputs such as counts, sums, and so on. Finally, the tool must perform those calculations in memory and produce the display the results.

These three elements are explored in order.

% ------------------------------------------------------------------------------------
% ### Calculation Table Config: Inputs, Files, Outputs
% ------------------------------------------------------------------------------------

\subsection{Specifying calculation table files, inputs, and outputs}

The most robust way to generate and display a table of numbers is by having the user develop their own post-processing scripts which output a simple CSV with the summary values and labels that they need. For this approach, nothing special is needed; whatever data analysis pipelines the analyst is already using are sufficient. Especially for more complex post-processing needs, using a high-quality platform such as Python or R is the recommended path.

For more simple summaries, we develop a way to extract and minimally process typical MATSim data files including CSV and XML formats. This is based on experience with the hesitancy of some analysts in our department to write Python and R scripts (since they are far more familiar with Java programming). If all one needs is the sum of a column of values or the count of some event types, learning Python and debugging a Python script is perhaps overkill.

As the file-based YAML configuration paradigm for SimWrapper is at this point well-established, a new YAML configuration schema for table calculations is the most natural way to express these table definitions.

A new \texttt{table-*.yml} YAML schema containing four sections emerged from extensive iteration with users. The four required sections are:

\begin{itemize}

  \item \textbf{files}: The set of input file or files required for the table. These can be raw MATSim outputs or the results of any pre- or post-processing that has already occurred in the analyst data pipeline.

  \item \textbf{interactive input entries}: This is a list of end-user-editable entries that are visible on interactive web form. Some calculations benefit from having a variable input which the user can specify; imagine fuel cost per liter, or number of vehicles in a taxi fleet. Each of these entries can have a default value.

  \item \textbf{calculations}: An ordered list of mathematical calculations to be performed. Variables, data columns, and interactive elements specified above are combined in equations as needed. This is described in more detail below.

  \item \textbf{outputs}: The final table entries are specified with formatting and labels.

\end{itemize}

Three of the four sections --- \emph{files}, \emph{interactive entries}, and \emph{outputs} --- are trivially specified and do not require extensive exposition here: they are straightforward definitions of file names, titles, and formatting directives. These are well-documented in the YY documentation available on the SimWrapper Website (\cite{SimWrapperWebsite}).

The calculations are explored next.

% ------------------------------------------------------------------------------------
% ### Calculation Table DSL: Specifying the Equations
% ------------------------------------------------------------------------------------
\subsection{Specifying calculations using a Domain-Specific Language}

The final and most important piece of specifying calculations is devising the equation format to be used in the YAML configuration, which brings together the interactive value inputs, the required input files, data columns in those files (and any data manipulations thereon), and combines them all in understandable algebraic equations that can be solved by the tool.

This is more akin to a "domain-specific language" (DSL) YY ACRONYM than a configuration file.

\cite{Visser2008} defines a domain-specific language as follows, "A domain-specific language (DSL) is a high-level software implementation language that supports concepts and abstractions that are related to a particular (application) domain." Visser explains further that a DSL is in essence "the encapsulation of design and implementation knowledge from a particular application or technical domain. The commonalities of the domain are implemented directly in a conventional programming language or indirectly in code generation templates, while the variability is configurable by the application developer through some configuration interface."

This is precisely what the YAML calculation definitions set out to do: allow a user who is an expert in the dataset and the needed transformations for a particular metric, to define that in a manner that doesn't require them to write a data analysis script in a full-fledged programming language such as R or Python.

YY continue

\subsection{ Construction of references}

References must be listed at the end of the paper. Do not begin them on a new page unless this is absolutely necessary. Authors should ensure that every reference in the text appears in the list of references and vice versa. Indicate references by \cite{Massimo2011} or \cite{Massimo2012} or \cite{Thomas2015} in the text.


\subsection{File naming and delivery}
Please title your files in this order `procedia acronym\_conference acronym\_authorslastname'.  Submit both the source file and the PDF to the Guest Editor.

Artwork filenames should comply with the syntax ``aabbbbbb.ccc'', where:
\begin{itemize}
\item a = artwork component type
\item b = manuscript reference code
\item c = standard file extension

Component types:
\item gr = figure
\item pl = plate
\item sc = scheme
\item fx = fixed graphic
\end{itemize}


% \subsection{Footnotes}
% Footnotes should be avoided if possible. Necessary footnotes should be denoted in the text by consecutive superscript letters\footnote{Footnote text.}. The footnotes should be typed single spaced, and in smaller type size (8 pt), at the foot of the page in which they are mentioned, and separated from the main text by a one line space extending at the foot of the column. The `Els-footnote' style is available in the ``TeX Template'' for the text of the footnote.

% Please do not change the margins of the template as this can result in the footnote falling outside printing range.


\begin{figure}[t]\vspace*{4pt}
%\centerline{\includegraphics{fx1}\hspace*{5mm}\includegraphics{fx1}}
\centerline{\includegraphics{gr1}}
\caption{(a) first picture; (b) second picture.}
\end{figure}


\section{Acknowledgements}
This research was funded in part by the German Federal Ministry of Transport and Digital Infrastructure (funding number 16AVF2160)

\bibliography{bibliography}
\bibliographystyle{../src/elsarticle-harv}

\end{document}

%%
%% End of file `procs-template.tex'.
